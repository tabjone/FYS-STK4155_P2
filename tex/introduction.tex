\section{Introduction}

%Done with structure

\begin{comment}
    In this report we will look at ...
    Motivate the reader, the first part of the introduction gives always a
    motivation and tries to give the overarching ideas. What I have done. 
    The structure of the report, how it is organised. Explain structure of the rapport at the end of intro. 
\end{comment}

In this report we will look at Gradient decent (GD) methods and Neural Networks
(NN) and it will be heavily based on the lecture notes by Morten Hjorth-Jensen
\cite{w41}.

GD methods are an effective way to find the local (and hopefully global)
minimum of a function. And it can not be understated how important GD methods
are in machine learning, statistics and optimization because of their simple
implementation and computational effectiveness.

Neural networks are a subset of machine learning based on artificial neurons,
weights and biases. These are important in many field and the use of NNs
are growing rapidly. They are especially useful in medicine as they can be
trained to recognize patterns in images much better than a human can. But we
will likely see neural networks implemented for a huge number of applications
in almost all industries in a couple of years. We will look at the benefits and
disadvantages of neural networks in different use cases such as polynomial
fitting and look at the optimal parameters for the network in those cases.
\\~\\
First we will introduce the OLS and Ridge method of regression, which will be
tested against the GD methods. Then we look at some of the most common GD methods. Namely plain
GD, Stochastic GD, GD with momentum, and RMSProp, AdaGrad and Adam. 

Then we move over to the neural network part of this report. We look at how to
initialize the network, the Feed Forward algorithm, different activation functions and
training using back propagation. We will then test the network on polynomial
fitting against regression methods.

After this we will test the network on a classification problem, namely
the Wisconsin breast cancer dataset provided by sklearn. And at this point we
will also implement logistic regression as this is suited for this particular
dataset. 
